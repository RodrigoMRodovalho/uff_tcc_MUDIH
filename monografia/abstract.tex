%%%%%%%%%%%%%%%%%%%%%%%%%%%%%%%%%%%%%%%%%%%%%%%%%%%%%%
%                      Abstract                      %
%%%%%%%%%%%%%%%%%%%%%%%%%%%%%%%%%%%%%%%%%%%%%%%%%%%%%%

\chapter*{Abstract}
\addcontentsline{toc}{chapter}{Abstract}

The multi-label learning aims to build classifiers that label, with more than one label, cases not labeled yet, such as diagnostic fails in a device, or genres of a music. An important issue of multi-label learning is related to the large number of examples (cases of learning) available, each instance associated with a few labels, and these, in turn, are derived from a large set of possible labels. The objective of this paper is to explore a non-supervised hierarchical learning method in order to improve the multi-label learning process, as a division and conquer technical of the problem. To achieve this goal, we will use the Mulan and Weka tools to support the development of methods being proposed, and will use natural databases to evaluate the performance of the method being proposed.\\

Keywords: Data Mining. Machine Learning.
