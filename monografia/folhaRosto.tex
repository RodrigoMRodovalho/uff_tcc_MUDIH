%%%%%%%%%%%%%%%%%%%%%%%%%%%%%%%%%%%%%%%%%%%%%%%%%%%%%%
%                   Folha de Rosto                   %
%%%%%%%%%%%%%%%%%%%%%%%%%%%%%%%%%%%%%%%%%%%%%%%%%%%%%%


\begin{center}

RODRIGO MAGALHÃES RODOVALHO

\vfill

USO DE UM MÉTODO DE APRENDIZADO NÃO-SUPERVISIONADO HIERÁRQUICO PARA MELHORAR A CONSTRUÇÃO
DE CLASSIFICADORES MULTIRRÓTULO PARA BASES DE DADOS COM GRANDE QUANTIDADE DE RÓTULOS.

\vspace{3.0cm}

\begin{flushright}
\begin{minipage}{0.45\textwidth}

Monografia apresentada ao Curso de \linebreak Bacharelado em Ciência da Computação da \linebreak Universidade Federal Fluminense, como \linebreak requisito parcial para obtenção do Grau de Bacharel. Área de Concentração: Mineração de Dados e Inteligencia Artificial.

\end{minipage}
\end{flushright}

\vspace{3.0cm}

Orientador: Prof. Dr. FLÁVIA CRISTINA BERNARDINI

\vfill

Niterói-RJ\\2016

\end{center}

\newpage
