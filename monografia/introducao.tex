%%%%%%%%%%%%%%%%%%%%%%%%%%%%%%%%%%%%%%%%%%%%%%%%%%%%%%%%
%                        Texto                         %
%%%%%%%%%%%%%%%%%%%%%%%%%%%%%%%%%%%%%%%%%%%%%%%%%%%%%%%%

%%%%%%%%%%%%%%%%%%%%%%%%%%%%%%%%%%%%%%%%%%%%%%%%%%%%%%%%
%                        Introdução                        %
%%%%%%%%%%%%%%%%%%%%%%%%%%%%%%%%%%%%%%%%%%%%%%%%%%%%%%%%

\chapter{Introdução}
\thispagestyle{empty} % retira numeracao da pagina, conforme as normas de apresentacao.
\label{chapter:intro}

Aprendizado multirrótulo é uma linha de pesquisa da sub-área de aprendizado de máquina com bastante foco nos últimos tempos. O aprendizado multirrótulo objetiva a construção de classificadores que rotulam, com mais de um rótulo, casos ainda não rotulados, como é o caso de diagnóstico de falhas em um equipamento, ou gêneros musicais de uma música. Uma questão importante do aprendizado multirrótulo está relacionado à grande quantidade de exemplos (casos de aprendizado) disponíveis com poucos rótulos associados, em geral, oriundo de um grande conjunto de rótulos
possíveis. O objetivo deste trabalho é investigar o uso de aprendizado de máquina não supervisionado para auxiliar o processo de aprendizado multirrótulo.

[colocar mais coisa]

No capítulo 2, são apresentados os conceitos de aprendizado de máquina. Assim como suas divisões e suas abordagens, aprendizado supervisionado,não supervisionado e aprendizado monorrótulo e multirrótulo, respectivamente. No capítulo 3, é apresentado uma proposta de um método de aprendizado multirrótulo baseado em aprendizado de máquina não-supervisionado hierárquico. No capítulo 4, são mostrados os resultados obtidos pela execução do método proposto e comparações de resulatos com um algoritmos conhecidos da literatura. No capítulo 5, é apresentado uma análise dos resultados obtidos e no capítulo 6, sugestões para trabalhos futuros.

