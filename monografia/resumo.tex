%%%%%%%%%%%%%%%%%%%%%%%%%%%%%%%%%%%%%%%%%%%%%%%%%%%%
%            Resumo na língua vernácula            %
%%%%%%%%%%%%%%%%%%%%%%%%%%%%%%%%%%%%%%%%%%%%%%%%%%%%

\chapter*{Resumo}
\addcontentsline{toc}{chapter}{Resumo}

O aprendizado multirrótulo tem por objetivo a construção de classificadores que rotulam, com mais de um rótulo, casos ainda não rotulados, como é o caso de diagnóstico de falhar em um equipamento, ou gêneros musicais de uma música. Uma questão importante do aprendizado multirrótulo está relacionado à grande quantidade de exemplos (casos de aprendizado) disponíveis, sendo cada exemplo associado a poucos rótulos, e esses, por sua vez, são oriundos de um grande conjunto de rótulos possíveis. O objetivo deste trabalho é explorar um método de aprendizado não-supervisionado hierárquico como forma de melhorar o processo de aprendizado multirrótulo, como uma técnica de divisão e conquista do problema. 
Para atingir esse objetivo, utilizaremos as ferramentas Mulan e Weka para apoiar o desenvolvimento do método a ser proposto, e utilizaremos bases de dados naturais para avaliar o desempenho do método a ser proposto.\\

Palavras-chave: Mineração de Dados. Aprendizado de Máquina. 
