%%%%%%%%%%%%%%%%%%%%%%%%%%%%%%%%%%%%%%%%%%%
	% Tópicos Diversos
%%%%%%%%%%%%%%%%%%%%%%%%%%%%%%%%%%%%%%%%%%%


\chapter{Tópicos diversos do Desenvolvimento}
\thispagestyle{empty} % retira numeracao da pagina, conforme as normas de apresentacao.

\section{Citações longas}

\begin{flushright}
\begin{minipage}{0.75\textwidth} % 75% de 160; margem para citações longas
\begin{quotation}
``O importante é não parar de questionar. A curiosidade tem sua própria razão
para existir. Uma pessoa não pode deixar de se sentir reverente ao contemplar
os mistérios da eternidade, da vida, da maravilhosa estrutura da realidade. Basta
que a pessoa tente apenas compreender um pouco mais desse mistério a cada dia.
Nunca perca uma sagrada curiosidade". Albert Einstein
\end{quotation}
\end{minipage}
\end{flushright}

\section{Tabelas}

\begin{table}[h]
	\centering
		\begin{tabular}{|c|c|c|}
		  \hline
			A & B & A + B \\ \hline
			0 & 0 & 0 \\ \hline
			0 & 1 & 1 \\ \hline
			1 & 0 & 1 \\ \hline
			1 & 1 & 1 \\ \hline
		\end{tabular}
		\caption{Tabela verdade para porta OR.}
\end{table}

\begin{table}[h]
	\centering
		\begin{tabular}{|c|c|c|}
		  \hline
			A & B & $\overline{A + B}$ \\ \hline
			0 & 0 & 1 \\ \hline
			0 & 1 & 0 \\ \hline
			1 & 0 & 0 \\ \hline
			1 & 1 & 0 \\ \hline
		\end{tabular}
		\caption{Tabela verdade para porta NOR.}
\end{table}

\section{Figuras}

Para inserir figuras deve-se colocar no preâmbulo o pacote graphicx e depois usar o comando
que permite inserir figura.

Por exemplo:

\begin{verbatim}
  \begin{figure}[h]
    \centering
    \includegraphics[width=4cm, height=6cm, angle=30]{grafico1.jpg}\\
    \caption{Título da figura.}
    \label{fig:NomeDeReferenciaParaAFigura}
  \end{figure}
\end{verbatim}

\section{Referências}

Abaixo, 
são apresentados alguns exemplos simples de referências bibliográficas.
Uma referência completa sobre todos os tipos de citações pode ser encontrada
em \cite{norma:esjo2005}.

Exemplos:

Exemplo de referência simples \cite{livro:circuit}.

Exemplo de referência dupla \cite{livro:circuit} e \cite{book:signal}.

Exemplo de referência múltipla ($> 2$) \cite{livro:circuit}--\cite{book:signal}.